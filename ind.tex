\documentclass[10pt]{article}
\usepackage{amssymb} %maths
\usepackage{amsmath} %maths
\usepackage[utf8]{inputenc} %useful to type directly diacritic characters
\usepackage[margin=2.6cm]{geometry}

\newcommand{\ind}[5]{\texttt{ind}(#1, \texttt{zero} \mapsto #2, (\texttt{succ}(#3), #4) \mapsto #5)}

\title{Indexed natural numbers}
\author{Dominic Orchard}

\begin{document}
\maketitle

\noindent
The standard essential details of syntax, operational semantics, and typing for indexed natural numbers.

\subsubsection*{Type syntax}
\begin{align}
\tag{type-level naturals}
N & ::= \mathbb{N} \mid N + N' \mid N * N' \mid \alpha \\
\tag{types} 
\tau & ::= \mathsf{Nat}\ N \mid \alpha \mid \ldots 
\end{align}

\subsubsection*{Term syntax}
\begin{align*}
e ::= \texttt{zero} \mid \texttt{succ}(e) \mid \ind{e}{e'}{x}{y}{e''}
\end{align*}
The first two are constructors. The last construct is the induction principle for eliminating
natural numbers where the first expression is the natural number to eliminate,
the second part is for handling the zero case, and the last is for handling
the successor case where $y$ gets bound to the recursive result.

\subsubsection*{Operational semantics (call-by-value)}

Values for natural numbers:
\begin{align*}
n ::= \texttt{zero} \mid \texttt{succ}(n)
\end{align*}
%
Reductions:
\begin{align*}
& \ind{\texttt{zero}}{e_0}{x}{y}{e_1} \leadsto e_0
\\[2em]
& \ind{\texttt{succ}(n)}{e_0}{x}{y}{e_1} \leadsto e_1[n/x][\ind{n}{e_0}{x}{y}{e_1}/y]
\\[2em]
& \dfrac{e \leadsto e'}{\ind{e}{e_0}{x}{y}{e_1} \leadsto \ind{e}{e_0}{x}{y}{e_1}}
\end{align*}

\subsubsection*{Typing rules}
\begin{align*}
\begin{array}{c}
\dfrac{}{\emptyset \vdash \texttt{zero} : \mathsf{Nat}\ 0}
\\[2em]
\dfrac{\Gamma \vdash e : \mathsf{Nat}\ n}{\emptyset \vdash \texttt{succ}(e) : \mathsf{Nat}\ (n+1)}
\\[2em]
\dfrac{\begin{array}{l}
       \Gamma \vdash e : \mathsf{Nat}\ n \\ \Gamma \vdash e_0 : \tau[0/\alpha] \\
        \Gamma, x : \mathsf{Nat}\ m, y : \tau[m/\alpha] \vdash e_1 : \tau[(m+1)/\alpha]\end{array}}
      {\Gamma \vdash \ind{e}{e_0}{x}{y}{e_1} : \tau[n/\alpha] }
\end{array}
\end{align*}

\end{document}
